% Options for packages loaded elsewhere
\PassOptionsToPackage{unicode}{hyperref}
\PassOptionsToPackage{hyphens}{url}
%
\documentclass[
]{article}
\usepackage{lmodern}
\usepackage{amssymb,amsmath}
\usepackage{ifxetex,ifluatex}
\ifnum 0\ifxetex 1\fi\ifluatex 1\fi=0 % if pdftex
  \usepackage[T1]{fontenc}
  \usepackage[utf8]{inputenc}
  \usepackage{textcomp} % provide euro and other symbols
\else % if luatex or xetex
  \usepackage{unicode-math}
  \defaultfontfeatures{Scale=MatchLowercase}
  \defaultfontfeatures[\rmfamily]{Ligatures=TeX,Scale=1}
\fi
% Use upquote if available, for straight quotes in verbatim environments
\IfFileExists{upquote.sty}{\usepackage{upquote}}{}
\IfFileExists{microtype.sty}{% use microtype if available
  \usepackage[]{microtype}
  \UseMicrotypeSet[protrusion]{basicmath} % disable protrusion for tt fonts
}{}
\makeatletter
\@ifundefined{KOMAClassName}{% if non-KOMA class
  \IfFileExists{parskip.sty}{%
    \usepackage{parskip}
  }{% else
    \setlength{\parindent}{0pt}
    \setlength{\parskip}{6pt plus 2pt minus 1pt}}
}{% if KOMA class
  \KOMAoptions{parskip=half}}
\makeatother
\usepackage{xcolor}
\IfFileExists{xurl.sty}{\usepackage{xurl}}{} % add URL line breaks if available
\IfFileExists{bookmark.sty}{\usepackage{bookmark}}{\usepackage{hyperref}}
\hypersetup{
  pdftitle={Chinook Marine Distribution Supplement--Implications of Management Actions},
  hidelinks,
  pdfcreator={LaTeX via pandoc}}
\urlstyle{same} % disable monospaced font for URLs
\usepackage[margin=1in]{geometry}
\usepackage{graphicx,grffile}
\makeatletter
\def\maxwidth{\ifdim\Gin@nat@width>\linewidth\linewidth\else\Gin@nat@width\fi}
\def\maxheight{\ifdim\Gin@nat@height>\textheight\textheight\else\Gin@nat@height\fi}
\makeatother
% Scale images if necessary, so that they will not overflow the page
% margins by default, and it is still possible to overwrite the defaults
% using explicit options in \includegraphics[width, height, ...]{}
\setkeys{Gin}{width=\maxwidth,height=\maxheight,keepaspectratio}
% Set default figure placement to htbp
\makeatletter
\def\fps@figure{htbp}
\makeatother
\setlength{\emergencystretch}{3em} % prevent overfull lines
\providecommand{\tightlist}{%
  \setlength{\itemsep}{0pt}\setlength{\parskip}{0pt}}
\setcounter{secnumdepth}{-\maxdimen} % remove section numbering

\title{Chinook Marine Distribution Supplement--Implications of Management
Actions}
\author{}
\date{\vspace{-2.5em}}

\begin{document}
\maketitle

Since 2008 Fisheries and Oceans Canada (DFO) has implemented a variety
of fisheries restrictions to minimize harvest impacts on stream-type
Fraser River Chinook salmon. The actions have included mark-selective
fisheries, slot limits, and spatio-temporal closures for retention, and
were specifically targeted at Spring 4.2, Spring 5.2, and Summer 5.2
management units (Dobson et al.~2020). Restrictions, even within a given
catch region, are not uniform, but vary among years, weeks, and
statistical areas. Here we consider how these restrictions may have
impacted our conclusions on stock composition and abundance. Although
the restrictions focused on marine and freshwater fisheries throughout
British Columbia, their impacts to our analysis are most applicable to
portions of the Juan de Fuca Strait and southern Strait of Georgia
regions. Thus we exclude other regions in southern BC. The scope of the
fisheries restrictions also expanded substantially in 2019; however, we
do not consider these latter adjustments here since they impact only one
data year.

Most broadly, fishery restrictions could impact our results by
decoupling estimates of stock-specific standardized CPUE (a
fisheries-dependent index) from true abundance. A technical review of
DFO's fisheries restrictions suggests that broadly they reduced harvest
impacts on stream-type Chinook salmon across multiple sectors; however,
their impact on catches in Juan de Fuca Strait recreational fisheries
specifically ranged from negligible to considerable (14\% increase to
48\% decrease relative to a reference period), depending on the year and
management unit (equivalent data in the southern Strait of Georgia were
not available; Dobson et al.~2020). Thus it appears likely, but not
certain, that abundance could be underestimated when fisheries
restrictions are in place, relative to when they are not. Ideally bias
in estimates of abundance could be quantified using fisheries
independent data; however, comprehensive sampling of Fraser River
Chinook salmon is restricted to freshwater test fisheries.

Fisheries restrictions may also impact estimates of composition from our
model independently of abundance if genetic samples become less
representative of catch in an area. Due to management restrictions on
retention of unclipped (presumed wild-origin) and large (presumed likely
to be stream-type) individuals during March to July, fish that are
considered ``legal'' in other areas are more likely to be released in
Juan de Fuca and southern Strait of Georgia fisheries. Although we
included genetic samples from such released individuals in our analysis,
these samples are relatively rare because they cannot be collected by
dockside creel observers and volunteers appear less likely to collect or
submit them.

To qualitatively assess how well genetic sampling matched estimated
catches we grouped Juan de Fuca Strait and southern Strait of Georgia
genetic and catch data into three management categories reflecting
relative exposure to fisheries restrictions (and accounting for temporal
and spatial variation in the application of restrictions). The first
category, ``status quo'', refers to data belonging to a spatial
(statistical area within a catch region) or temporal strata to which the
fisheries restrictions do not apply. The second category, ``exempted'',
includes data from catch taken in spatio-temporal strata where
restrictions occurred, but which met those mark or size restrictions
(i.e., ``legal'' catch, such as adipose clipped fish caught in a strata
with mark-selective fisheries). In the case of catch data, exempted data
are \emph{landed} catch estimates (estimated from creel surveys), which
we assume met mark or size restrictions. In the case of composition
data, exempted data were samples collected from fish that met mark or
size restrictions. The third category, ``restricted'', includes data
from catch taken in spatio-temporal strata where restrictions occurred
which do not meet mark or size restrictions (or where blanket
non-retention measures were in place). For catch data, restricted data
are catch estimates for \emph{released}, legal-sized fish (again,
estimated from creel surveys). For composition data, restricted data are
samples collected from fish that were released or from fish that were
landed, but did not meet mark or size restrictions (i.e., should have
been released). We note that these are relatively ad hoc comparisons and
several assumptions apply. For example, landed catch estimates likely
include fish that should have been released and released catch estimates
likely include fish that met mark or size restrictions, but were not
retained for other reasons.

To simplify comparisons we excluded months where the relevant
restrictions never applied. Catch data were not available for all
statistical areas in March-May for Juan de Fuca Strait (see Figure S1 in
main text for details), resulting in no status quo catches for these
months and their exclusion from estimates of stock-specific standardized
CPUE in the main text. The partial data were included in this
comparison, however, because all GSI samples for those months originated
from the same statistical area.

Catches from spatio-temporal strata with restrictions in place were the
greatest proportion of total catch early in the year, declining after
May in Juan de Fuca Strait and April in the southern Strait of Georgia
as restrictions were removed or catches began to incorporate
unrestricted statistical areas (Figure 1, green and yellow bars). While
the majority of the catch during these months was categorized as
exempted, suggesting most individuals met size and mark related
restrictions, restricted catches were still considerable. This was
particularly true in 2019 when Juan de Fuca management restrictions were
most severe.

\begin{figure}
\centering
\includegraphics{management_restrictions_supp_files/figure-latex/catch-plot-1.pdf}
\caption{Proportion of catch assigned to management categories based on
year-specific fisheries restrictions (defined in text) showing
relatively large proportion of released catch early in year, as well as
minimal catch data from areas with status quo management restrictions
until mid-summer. Only months and regions impacted by fisheries
restrictions are shown. Blank spaces represent strata without data.}
\end{figure}

Although samples from restricted individuals were collected, they made
up a relatively small proportion of the total for these catch regions
(Figure 2). Coverage was particularly poor in the southern Strait of
Georgia in March and April, where the majority of samples originated in
statistical areas that were not impacted by the restrictions, while
catch estimates originated in statistical areas where they did. Juan de
Fuca Strait samples predominantly originated in similar strata to the
catch; however, they largely consisted of samples from individuals that
were exempted from the restrictions.

\begin{figure}
\centering
\includegraphics{management_restrictions_supp_files/figure-latex/composition-plot-1.pdf}
\caption{Proportion of genetic stock indetification samples assigned to
management categories based on year-specific fisheries restrictions
(defined in text). Note minimal sampling of released catches and large
contribution of status quo areas in southern Strait of Georgia to
estimates. Only months and regions impacted by fisheries restrictions
are shown. Blank spaces represent strata without data.}
\end{figure}

The small number of genetic samples from individuals in the restricted
management category, relative to their proportional contribution to
catch, suggests that certain stocks may be underestimated in composition
predictions. Presumably this is most likely for Fraser River early run
stocks, which contain the majority of stream-type populations of
conservation concern. It is difficult to evaluate the magnitude of these
effects without additional data; however, we chose to model composition
at the scale of catch regions, not statistical areas, precisely due to
limited data at finer spatial scales. Effectively this approach assumes
that composition is homogeneous within a catch region and should mute
differences between estimated and true composition by integrating data
from multiple statistical areas, only a subset of which are impacted by
restrictions. While such an assumption is likely untrue, it is not
possible to robustly evaluate, or remedy, without a more comprehensive
genetic sampling program.

Despite these issues, our predictions of stock-specific standardized
CPUE appear consistent with data from Fraser River freshwater test
fisheries, the least biased and longest running data available on adult
migration timing. In both datasets, predicted peak Fraser Spring 4.2
abundance in Juan de Fuca Strait is immediately followed by the Spring
5.2 peak, followed by peaks in Summer 5.2 and Summer 4.1 abundance about
a month later (Figure S14; Dobson et al.~2020). Furthermore, the
relative abundance of each of these stock groups is approximately
equivalent to the proportional mean return of each to Fraser River test
fisheries. Thus even if the contribution of these stocks is
underestimated in certain statistical areas, it appears unlikely to
influence our general conclusions. Nevertheless, to inform future
fisheries management decisions, composition and catch data should be
matched at the finest resolution possible.

\hypertarget{literature-cited}{%
\subsubsection{Literature Cited}\label{literature-cited}}

Dobson, D., Holt, K., and Davis, B. 2020. A technical review of the
management approach for stream-type Fraser River Chinook. DFO Can. Sci.
Advis. Sec. Res. Doc. 2020/027. x. + 280 p.

\end{document}
