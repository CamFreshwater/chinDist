\documentclass[]{article}
\usepackage{lmodern}
\usepackage{amssymb,amsmath}
\usepackage{ifxetex,ifluatex}
\usepackage{fixltx2e} % provides \textsubscript
\ifnum 0\ifxetex 1\fi\ifluatex 1\fi=0 % if pdftex
  \usepackage[T1]{fontenc}
  \usepackage[utf8]{inputenc}
\else % if luatex or xelatex
  \ifxetex
    \usepackage{mathspec}
  \else
    \usepackage{fontspec}
  \fi
  \defaultfontfeatures{Ligatures=TeX,Scale=MatchLowercase}
\fi
% use upquote if available, for straight quotes in verbatim environments
\IfFileExists{upquote.sty}{\usepackage{upquote}}{}
% use microtype if available
\IfFileExists{microtype.sty}{%
\usepackage{microtype}
\UseMicrotypeSet[protrusion]{basicmath} % disable protrusion for tt fonts
}{}
\usepackage[margin=1in]{geometry}
\usepackage{hyperref}
\hypersetup{unicode=true,
            pdftitle={WCVI Troll Catch},
            pdfborder={0 0 0},
            breaklinks=true}
\urlstyle{same}  % don't use monospace font for urls
\usepackage{color}
\usepackage{fancyvrb}
\newcommand{\VerbBar}{|}
\newcommand{\VERB}{\Verb[commandchars=\\\{\}]}
\DefineVerbatimEnvironment{Highlighting}{Verbatim}{commandchars=\\\{\}}
% Add ',fontsize=\small' for more characters per line
\usepackage{framed}
\definecolor{shadecolor}{RGB}{248,248,248}
\newenvironment{Shaded}{\begin{snugshade}}{\end{snugshade}}
\newcommand{\AlertTok}[1]{\textcolor[rgb]{0.94,0.16,0.16}{#1}}
\newcommand{\AnnotationTok}[1]{\textcolor[rgb]{0.56,0.35,0.01}{\textbf{\textit{#1}}}}
\newcommand{\AttributeTok}[1]{\textcolor[rgb]{0.77,0.63,0.00}{#1}}
\newcommand{\BaseNTok}[1]{\textcolor[rgb]{0.00,0.00,0.81}{#1}}
\newcommand{\BuiltInTok}[1]{#1}
\newcommand{\CharTok}[1]{\textcolor[rgb]{0.31,0.60,0.02}{#1}}
\newcommand{\CommentTok}[1]{\textcolor[rgb]{0.56,0.35,0.01}{\textit{#1}}}
\newcommand{\CommentVarTok}[1]{\textcolor[rgb]{0.56,0.35,0.01}{\textbf{\textit{#1}}}}
\newcommand{\ConstantTok}[1]{\textcolor[rgb]{0.00,0.00,0.00}{#1}}
\newcommand{\ControlFlowTok}[1]{\textcolor[rgb]{0.13,0.29,0.53}{\textbf{#1}}}
\newcommand{\DataTypeTok}[1]{\textcolor[rgb]{0.13,0.29,0.53}{#1}}
\newcommand{\DecValTok}[1]{\textcolor[rgb]{0.00,0.00,0.81}{#1}}
\newcommand{\DocumentationTok}[1]{\textcolor[rgb]{0.56,0.35,0.01}{\textbf{\textit{#1}}}}
\newcommand{\ErrorTok}[1]{\textcolor[rgb]{0.64,0.00,0.00}{\textbf{#1}}}
\newcommand{\ExtensionTok}[1]{#1}
\newcommand{\FloatTok}[1]{\textcolor[rgb]{0.00,0.00,0.81}{#1}}
\newcommand{\FunctionTok}[1]{\textcolor[rgb]{0.00,0.00,0.00}{#1}}
\newcommand{\ImportTok}[1]{#1}
\newcommand{\InformationTok}[1]{\textcolor[rgb]{0.56,0.35,0.01}{\textbf{\textit{#1}}}}
\newcommand{\KeywordTok}[1]{\textcolor[rgb]{0.13,0.29,0.53}{\textbf{#1}}}
\newcommand{\NormalTok}[1]{#1}
\newcommand{\OperatorTok}[1]{\textcolor[rgb]{0.81,0.36,0.00}{\textbf{#1}}}
\newcommand{\OtherTok}[1]{\textcolor[rgb]{0.56,0.35,0.01}{#1}}
\newcommand{\PreprocessorTok}[1]{\textcolor[rgb]{0.56,0.35,0.01}{\textit{#1}}}
\newcommand{\RegionMarkerTok}[1]{#1}
\newcommand{\SpecialCharTok}[1]{\textcolor[rgb]{0.00,0.00,0.00}{#1}}
\newcommand{\SpecialStringTok}[1]{\textcolor[rgb]{0.31,0.60,0.02}{#1}}
\newcommand{\StringTok}[1]{\textcolor[rgb]{0.31,0.60,0.02}{#1}}
\newcommand{\VariableTok}[1]{\textcolor[rgb]{0.00,0.00,0.00}{#1}}
\newcommand{\VerbatimStringTok}[1]{\textcolor[rgb]{0.31,0.60,0.02}{#1}}
\newcommand{\WarningTok}[1]{\textcolor[rgb]{0.56,0.35,0.01}{\textbf{\textit{#1}}}}
\usepackage{graphicx,grffile}
\makeatletter
\def\maxwidth{\ifdim\Gin@nat@width>\linewidth\linewidth\else\Gin@nat@width\fi}
\def\maxheight{\ifdim\Gin@nat@height>\textheight\textheight\else\Gin@nat@height\fi}
\makeatother
% Scale images if necessary, so that they will not overflow the page
% margins by default, and it is still possible to overwrite the defaults
% using explicit options in \includegraphics[width, height, ...]{}
\setkeys{Gin}{width=\maxwidth,height=\maxheight,keepaspectratio}
\IfFileExists{parskip.sty}{%
\usepackage{parskip}
}{% else
\setlength{\parindent}{0pt}
\setlength{\parskip}{6pt plus 2pt minus 1pt}
}
\setlength{\emergencystretch}{3em}  % prevent overfull lines
\providecommand{\tightlist}{%
  \setlength{\itemsep}{0pt}\setlength{\parskip}{0pt}}
\setcounter{secnumdepth}{0}
% Redefines (sub)paragraphs to behave more like sections
\ifx\paragraph\undefined\else
\let\oldparagraph\paragraph
\renewcommand{\paragraph}[1]{\oldparagraph{#1}\mbox{}}
\fi
\ifx\subparagraph\undefined\else
\let\oldsubparagraph\subparagraph
\renewcommand{\subparagraph}[1]{\oldsubparagraph{#1}\mbox{}}
\fi

%%% Use protect on footnotes to avoid problems with footnotes in titles
\let\rmarkdownfootnote\footnote%
\def\footnote{\protect\rmarkdownfootnote}

%%% Change title format to be more compact
\usepackage{titling}

% Create subtitle command for use in maketitle
\providecommand{\subtitle}[1]{
  \posttitle{
    \begin{center}\large#1\end{center}
    }
}

\setlength{\droptitle}{-2em}

  \title{WCVI Troll Catch}
    \pretitle{\vspace{\droptitle}\centering\huge}
  \posttitle{\par}
    \author{}
    \preauthor{}\postauthor{}
    \date{}
    \predate{}\postdate{}
  

\begin{document}
\maketitle

Tissues samples were collected from WCVI commercial troll vessels
(2007-2015) and assigned to individual stocks using DFO genetics lab's
coastwide baseline. Samples were collected in bulk vials (i.e.~multiple
muscle plugs per vial) and currently can be assigned to catch region
(north or south westcoast VI) and month, but not to statistical
area/troll zone or week/day (additional data could be interpolated using
FOS database).

Sampling effort was considerable and approximately proportional to
catch. Sampling effort for stock ID (relative to catch) averaged NA
(range NA, NA).

\begin{Shaded}
\begin{Highlighting}[]
\NormalTok{catchAgg <-}\StringTok{ }\NormalTok{catch }\OperatorTok\StringTok{ }
\StringTok{  }\KeywordTok{group_by}\NormalTok{(catchReg, month, year, labN) }\OperatorTok\StringTok{ }
\StringTok{  }\KeywordTok{mutate}\NormalTok{(}\DataTypeTok{sumCatch =} \KeywordTok{sum}\NormalTok{(estCatch)) }\OperatorTok\StringTok{ }
\StringTok{  }\KeywordTok{select}\NormalTok{(catchReg, month, year, labN, sumEffort, sumCatch) }\OperatorTok\StringTok{ }
\StringTok{  }\KeywordTok{mutate}\NormalTok{(}\DataTypeTok{cpue =}\NormalTok{ sumCatch }\OperatorTok{/}\StringTok{ }\NormalTok{sumEffort) }\OperatorTok\StringTok{ }
\StringTok{  }\KeywordTok{distinct}\NormalTok{()}

\KeywordTok{ggplot}\NormalTok{(catchAgg) }\OperatorTok{+}
\StringTok{  }\KeywordTok{geom_bar}\NormalTok{(}\KeywordTok{aes}\NormalTok{(}\DataTypeTok{x =} \KeywordTok{as.factor}\NormalTok{(month), }\DataTypeTok{y =}\NormalTok{ labN, }\DataTypeTok{fill =}\NormalTok{ catchReg), }
       \DataTypeTok{stat =} \StringTok{"identity"}\NormalTok{) }\OperatorTok{+}
\StringTok{    }\KeywordTok{scale_fill_viridis_d}\NormalTok{() }\OperatorTok{+}
\StringTok{  }\KeywordTok{labs}\NormalTok{(}\DataTypeTok{x =} \StringTok{""}\NormalTok{, }\DataTypeTok{y =} \StringTok{"GSI Samples"}\NormalTok{, }\DataTypeTok{fill =} \StringTok{"Catch Region"}\NormalTok{) }\OperatorTok{+}
\StringTok{  }\KeywordTok{facet_wrap}\NormalTok{(}\OperatorTok{~}\NormalTok{year)}
\end{Highlighting}
\end{Shaded}

\includegraphics{wcviTrollSummary_files/figure-latex/sampleSize-1.pdf}

Although the west coast troll fishery currently occurs in August and
early September (constrained by interior Fraser coho initially, then
steelhead), until recently effort peaked in May. Conversely CPUE,
ignoring stock composition, peaked later in summer (July/August)
depending on catch region.

\begin{Shaded}
\begin{Highlighting}[]
\NormalTok{eff <-}\StringTok{ }\KeywordTok{ggplot}\NormalTok{(catchAgg, }\KeywordTok{aes}\NormalTok{(}\DataTypeTok{x =} \KeywordTok{as.factor}\NormalTok{(month), }\DataTypeTok{y =}\NormalTok{ sumEffort, }
                            \DataTypeTok{fill =}\NormalTok{ catchReg)) }\OperatorTok{+}
\StringTok{  }\KeywordTok{scale_fill_viridis_d}\NormalTok{() }\OperatorTok{+}
\StringTok{  }\KeywordTok{geom_boxplot}\NormalTok{() }\OperatorTok{+}
\StringTok{  }\KeywordTok{labs}\NormalTok{(}\DataTypeTok{x =} \StringTok{""}\NormalTok{, }\DataTypeTok{y =} \StringTok{"Monthly Effort (boat days)"}\NormalTok{, }\DataTypeTok{color =} \StringTok{"Catch Region"}\NormalTok{)}
\NormalTok{cpue <-}\StringTok{ }\KeywordTok{ggplot}\NormalTok{(catchAgg, }\KeywordTok{aes}\NormalTok{(}\DataTypeTok{x =} \KeywordTok{as.factor}\NormalTok{(month), }\DataTypeTok{y =}\NormalTok{ cpue, }
                             \DataTypeTok{fill =}\NormalTok{ catchReg)) }\OperatorTok{+}
\StringTok{  }\KeywordTok{scale_fill_viridis_d}\NormalTok{() }\OperatorTok{+}
\StringTok{  }\KeywordTok{geom_boxplot}\NormalTok{() }\OperatorTok{+}
\StringTok{  }\KeywordTok{labs}\NormalTok{(}\DataTypeTok{x =} \StringTok{"Month"}\NormalTok{, }\DataTypeTok{y =} \StringTok{"CPUE (Chinook per boat day)"}\NormalTok{, }\DataTypeTok{fill =} \StringTok{"Catch Region"}\NormalTok{)}
\KeywordTok{ggarrange}\NormalTok{(eff, cpue, }\DataTypeTok{ncol =} \DecValTok{1}\NormalTok{, }\DataTypeTok{nrow =} \DecValTok{2}\NormalTok{, }\DataTypeTok{common.legend =} \OtherTok{TRUE}\NormalTok{)}
\end{Highlighting}
\end{Shaded}

\includegraphics{wcviTrollSummary_files/figure-latex/plotEffort-1.pdf}

Although samples cannot yet be attributed to specific stat areas,
catches can. Majority of effort in offshore areas (catch trends, not
shown, are similar). Areas 125 and 126 in the north, 123 in the south.
Note, however, that 123 effort has declined substantially in recent
years.

\begin{Shaded}
\begin{Highlighting}[]
\KeywordTok{ggplot}\NormalTok{(fosCatch, }\KeywordTok{aes}\NormalTok{(}\DataTypeTok{x =} \KeywordTok{as.factor}\NormalTok{(FISHING.MONTH), }\DataTypeTok{y =}\NormalTok{ VESSELS_OP, }
                             \DataTypeTok{fill =} \KeywordTok{as.factor}\NormalTok{(MGMT_AREA))) }\OperatorTok{+}
\StringTok{  }\KeywordTok{geom_boxplot}\NormalTok{() }\OperatorTok{+}
\StringTok{  }\KeywordTok{labs}\NormalTok{(}\DataTypeTok{x =} \StringTok{"Month"}\NormalTok{, }\DataTypeTok{y =} \StringTok{"Effort (boat days)"}\NormalTok{, }\DataTypeTok{fill =} \StringTok{"Management Area"}\NormalTok{) }\OperatorTok{+}
\StringTok{  }\KeywordTok{scale_fill_viridis_d}\NormalTok{() }\OperatorTok{+}
\StringTok{  }\KeywordTok{facet_wrap}\NormalTok{(}\OperatorTok{~}\NormalTok{CATCH_REGION)}
\end{Highlighting}
\end{Shaded}

\includegraphics{wcviTrollSummary_files/figure-latex/saCatch-1.pdf}

Relative catch contributions depend on how the data are aggregated
across time and rolled up regionally. For example absolute catches are,
on average, dominated by Columbia River fish in the north and Columbia
plus Puget Sound fish in the south.

\includegraphics{wcviTrollSummary_files/figure-latex/aggComp-1.pdf}

These patterns can be visualized more clearly with proportional catches.
Again Columbia River and Puget Sound fish are the dominant contributor,
however Fraser abundance peaks again in August.

\includegraphics{wcviTrollSummary_files/figure-latex/propAggComp-1.pdf}

Yet another way to visualize these data are to focus on how the seasonal
abundance of a specific aggregate changes through time. Here we see that
west coast Vancouver Island stocks tend to peak relatively early and
Fraser River stocks relatively late.

\begin{verbatim}
## Warning: Removed 20 rows containing non-finite values (stat_boxplot).
\end{verbatim}

\includegraphics{wcviTrollSummary_files/figure-latex/propTS-1.pdf}

Ignoring interannual variation (for now) we can also look at the
cumulative proportion of the catch through the calendar year. Note that
it may be reasonable to consider alternative time frames such as having
the year ``end'' when the last spawner should have passed through an
area.

\includegraphics{wcviTrollSummary_files/figure-latex/cumSum-1.pdf}

May be important to account for stock-specific variation within a
region. As an example focus on individual stocks within Fraser. There
are relatively large contributions of South Thompson fish throughout the
spring and summer, as well as noticeable contributions of Upper Fraser
fish during the summer, but only in NWVI catches. In SWVI almost
exclusively dominated by fall Fraser fish.

\includegraphics{wcviTrollSummary_files/figure-latex/fraser-1.pdf}

Plot cumulative catch, again ignoring interannual variablity, to provide
a coarse index of passage through the fishery.

\includegraphics{wcviTrollSummary_files/figure-latex/frPropTS-1.pdf}

\includegraphics{wcviTrollSummary_files/figure-latex/frCumSum-1.pdf}


\end{document}
